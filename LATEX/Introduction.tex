
\section{Introduction}
\label{section:Introduction}

This project develops novel methods that incorporate information from multiple imaging modalities for improved interpretation of biological processes.  The proposal is organized around two primary efforts:
%
1) First, novel methods are proposed that enhance tissue classification by incorporating complementary information from multiple modality scans with different spatial resolutions.  The approach allows information acquired from low-resolution scans (i.e. T2-weighted, PD-weighted, or diffusion-weighted imaging (DWI) data) to improve segmentation quality over previously produced results that utilized only high-resolution structural MR images.
%
2) Second, we propose a novel method for super-resolution reconstruction using the prior anatomical information extracted from high spatial resolution images to improve biological interpretation of low-resolution  modality sources.  For example, we incorporate the high biological resolution of brain anatomy description from structural MR (T1/T2-wighted) scans to improve the super-resolution reconstruction of DWI images from the same subject. The proposed method aims to increase the biological interpretability of input low-resolution DWI to a high-resolution DWI by a factor of 8 ($2.0^{3}mm$).

This research proposal focuses on challenges encountered in multi-site neurodegenerative studies like the PREDICT-HD \cite{PREDICTHD} study of Huntington's Disease (HD).  We describe novel approaches for the segmentation and super-resolution reconstruction (SRR), that help improving interpretability of data and can, for example, increase the sensitivity of volumetric and diffusion measures used by clinicians to improve the clinical trials that test therapies to slow down the progression of disease. Increasing the sensitivity of measures can also reduce the number of samples needed for longitudinal and cross-sectional analysis that can potentially reduce the cost of future studies. 

In similar multi-site longitudinal studies of neurodegenerative disorders such as Alzheimer's disease, Schizophrenia and Huntington's Disease, usually different modalities are acquired at each scanning session, since each modality scan can reveal different characteristics of underlying biological architecture. T1-weighted MR is the most used modality; however, it is very hard to distinguish all biological tissue types (e.g. CSF from the bone marrow, or the gray matter from the blood vessels) by looking only at T1 scan. Using complementary information from other modalities (e.g. T2-weighted scan) can provide better contrast in the case of above-mentioned T1-only tissue type ambiguities.

%% REMOVED DWI is another modality that maps water molecule diffusion patterns in biological tissues that can reveal abnormalities in white matter fiber structure and provide models of brain connectivity. Multiple 3D volumes are collected in a single 4D DWI scan, where each 3D volume quantifies diffusion in a different direction. The different sensitizing gradient directions are created by changing the direction of pulsed gradient fields. In a 4D DWI scan, a baseline image or $b0$ image without diffusion weighting is also needed to quantify the diffusion in the DWI data. 
%\newline

%\noindent 
%% REMOVED Diffusion-weighted MR imaging been shown to be sensitive to microscopic degeneration of white matter that is not measurable in structural MRI. Measuring diffusion properties of functional white matter areas in the time course of degeneration will help us to advance our understanding of neuropathological basis in neurodegenerative diseases.
%% REMOVED However, collected DWI data usually contain too much noise and many artifacts, that avoids them to be efficiently used in standard methods of analysis.
%\newline

%\noindent 
%% REMOVED This research proposal is motivated by the results of a recent study that developed an automated pipeline for the processing of diffusion-weighted imaging (DWI) data in a multi-modal study.
%\noindent The development of reliable and valid measurements in brain morphology and biological architecture is important in study of subtle differences in neurodegenerative disorders like Alzheimer's disease, schizophrenia and HD. To study these subtle differences, large samples of subjects from many data collection sites must be studied.
%% REMOVED In our previous studies, we have used neuroimaging in python pipelines and interfaces (NIPYPE) \cite{Nipype} to develop an extensible, automated processing pipeline 
%apply novel cutting-edge methods of \textit{in vivo} diffusion MRI analysis into large-scale multicenter studies. This pipeline provided a reliable tool to non-invasively study alternations in key brain white matter structures for cross-sectional and longitudinal analysis.
%% REMOVED to perform a volumetric analysis on a large scale heterogeneous DWI dataset from the PREDICT-HD database. This DWI pipeline consists of a series of image processing techniques assembled for artifact-correction, alignment, segmentation, compressed sensing, tensor and scalar maps estimation, and statistical measurements. This workflow extended the previous pre-processing steps \cite{Matsui2014} to provide an extensible infrastructure for reliable cross-sectional and longitudinal analysis that allows neuroscientists to measure information about white matter tissues previously inaccessible with standard methods of analysis.
%% REMOVED We investigated the ability of this NIPYPE wrapped pipeline to utilize High Performance Computing (HPC) resources to achieve time-efficient data processing and analysis for large-scale studies. The pipeline outputs provided a validated valuable set of information derived from diffusion-weighted imaging data. 

The first part of this work proposes novel methods to use complementary information from multliple modalities (including complementary summary information from DWI images) to enhance the classification results previously generated from high-resolution structural MRI data alone. The second part uses high biological-resolution information provided by structural MRI data for guiding super-resolution reconstruction of each of the 3D DWI subvolume gradients to provide more biologically interpretable DWI analysis results.
%\newline

\subsection{Aim 1: Enhance tissue classification by incorporating multiple modalities with different spatial resolutions}

%% REMOVED The successful implementation of recently developed DWI pipeline encouraged us to use its valuable output information to improve the segmentation quality for large-scale multi-site longitudinal MR images by incorporating complimentary information acquired from DWI data.
We show increased sensitivity and interpretability of data by carefully using information from multiple modality channels.  We propose a novel method to overcome the challenges of partial volume effects that are encountered from multimodal data (especially at different resolutions).  The challenges are described and compared against failures that occur when na\"{i}vely adding multiple modalities.%In this aim, we break down the problem and find the solution.
%\newline

\subsubsection{Subtask 1: Tissue classification of large-scale multi-site heterogeneous MR data using fuzzy k-nearest neighbor method}

This task aims to improve automated classification of brain tissues for multi-center 3D MRI data analysis. 
Previous studies have developed a robust multi-modal tool for automated registration, bias-field correction and tissue classification based on expectation maximization (EM) method \cite{Kim2013} that is group specific and uses a \emph{priori} knowledge for all the subjects in an atlas-based approach.
This task, however, emphasizes the importance of a non-parametric model's utility in neurodegenerative diseases, since each subject has unique anatomical states in longitudinal degenerative studies that may not be represented by prior probability distributions. Enhancements are suggested by augmenting the $EM$-based classification using a fuzzy k-nearest neighbor (k-NN) classifier that builds up a model for each individual subject and complements the classification results that $EM$ produces.
A Fuzzy $k-NN$ method was selected, as this non-parametric classifier is not biased by prior probability distributions, and it potentially can model more complex decision boundaries than a Gaussian distribution based mixture method.
%\newline

\subsubsection{Subtask 2: Enhance multi-modal classification when complementary information comes from a second modality with lower spatial resolution}

The second task investigates the partial volume effect (PVE) on multi-modal classification when one modality channel provides information in lower spatial resolution. Partial volume effect complicates the segmentation process, and, due to the complexity of human brain anatomy, addressing the $PVE$ issue is a critical factor for accurate brain structure classification from multiple modalities. Therefore, a novel method is proposed to identify spatial samples that are not affected by partial volume composition, termed as \textit{pure samples}, to train/initialize the classification methods.
%\newline

\subsubsection{Subtask 3: Use previously un-exploited imaging low resolution scan data for further enhancement of tissue classification}


Further potential improvements in tissue classification results will be investigated by incorporating the complementary enhanced white matter information acquired from diffusion-weighted imaging data into the enhanced classification framework. Here, we suggest to:
\begin{itemize}
    \item[-] Extract isotropic diffusion-weighted information (IDWI) from DWI dataset, that is the geometric mean of all diffusion images.
    \item[-] Incorporate the extracted IDWI (high contrast biological white matter) image into the segmentation process as a new modality channel.
\end{itemize}

%\noindent 
Diffusion information provides better contrast between white matter (WM) and gray matter (GM) tissue regions that is expected to enhance the WM/GM classification.

\subsection{Aim 2: Super-resolution reconstruction of low-resolution diffusion-weighted imaging data using \emph{a priori} knowledge of anatomical structure}

The first aim proposed a method to address partial volume effect, so we can use the low-resolution information including the complementary information derived from DWI data to enhance the segmentation results for the structural MR data.
The second aim, on the other hand, suggests to use the high-resolution representation of anatomical structures provided by the structural MR data as \emph{a priori} knowledge to enhance the resolution (in a biologically relevant manner) of diffusion-weighted imaging data.
% REMOVE  I don't know what this means: In fact, reconstruction of high-resolution images may also benefit from reducing the partial volume effects.

The use of diffusion-weighted imaging data is strongly limited by its relatively low spatial resolution. The resolution of a typical DWI data is $2mm \times 2mm \times 2mm$, that means its voxel size is $8 mm^{3}$ times larger than that in structural MRI.
The goal of this section is to increase the spatial resolution of an input DWI scan to match the anatomical resolution present in the isotropic ($1$ $mm^3$) voxel typically present in structural MR images. However, using only basic interpolation techniques to resample the input DWI data to a higher resolution lattice space introduces no new information or improvements to interpretation that can help further medical imaging analysis. This aim investigates the use of tissue boundary information from the high-resolution structural MRI data to achieve a super-resolution reconstruction of DWI scans, where the voxel values at tissue boundaries will be in more agreement with the actual anatomical definitions of morphological data.
\newline