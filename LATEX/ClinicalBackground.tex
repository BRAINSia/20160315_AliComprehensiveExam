\section{Clinical Motivation}
\label{clinicalBackground}

\subsection{Huntington's Disease}

\subsubsection{Disease summary}

\noindent Huntington's Disease (HD) is an autosomal dominant, neurodegenerative progressive disorder that cause degeneration and cell loss in specific brain areas and is characterized by its motor, cognitive, and behavioral disturbances \cite{PREDICTHD}. HD results from an expansion of trinucleotide repeat CAG in gene IT-15, that encodes the protein huntingtin on chromosome 4 \cite{Huntington1993} and results in widespread neuronal degeneration preferentially within the stratum \cite{Montoya2006}.

Diagnosis occurs at the emergence of motor symptoms usually in midlife (mean age of onset ranging from 35 to 42 years) \cite{Martin1986}, although structural and metabolic brain changes, neuropsychological deficits, and psychiatric symptoms often precede the neurologic disease manifestation in pre-symptomatic individuals who have been confirmed as carriers of HD gene mutation \cite{Paulsen2001, Campodonico1998, Aylward2000, Harris1999}. Disease duration lasts from 17 to as much as 30 years after motor symptom onset, where the duration is often shorter when the disease begins earlier in life \cite{Martin1986, Gomez-Tortosa2001}.

No cure is currently available for this disease, and all available treatments only target symptoms. There are no pharmacological solutions for slowing or stopping disease progression \cite{Frank2010}. The development of any potential treatment is dependent upon a detailed understanding of progression of HD throughout a patient's lifetime, so one of main thrusts in HD research is to investigate and characterize possible biological and clinical markers of disease progression in prodromal HD patients. In vivo neuroimaging provides an exciting opportunity to develop biomarkers for this disease progression and, most importantly, better target potential treatments that hopefully will slow or stop the progression of disease during the prodromal period, that is the time before symptom onset \cite{Paulsen2008}.
\newline

\subsubsection{PREDICT-HD Study}

The developed methods in present study will be performed and validated using a well-characterized large cohort of MRI scans from PREDICT-HD. The Neurobiological Predictors of Huntington’s Disease (PREDICT-HD) is an international 32-site observational study of longitudinal neurodegeneration of prodromal HD participants. The purpose of PREDICT-HD is to identify the earliest changes in the natural progression of disease before clinical diagnosis \cite{PREDICTHD}.

PREDICT-HD database contains structural MRI and DWI of brain and a comprehensive set of clinical correlates, that are non-imaging measures including motor and cognitive tests, from a very large sample of individuals with pre-manifest (prodromal) HD, whose HD symptomatology were tracing from its very earliest signs. To combine the strengths of PREDICT-HD dataset with cutting-edge neuroimaging technologies, it is important to have reliable methods that performs efficiently and consistently for data from many collection sites.

Findings from the PREDICT-HD study are consistent, and they demonstrate volume reduction in the basal ganglia and in particular the striatum, as well as a reduction of cerebral white matter volume \cite{Paulsen2008a, Rosas2001}, cerebral cortex size \cite{Paulsen2006} and morphology \cite{Nopoulos2010}. Also, smaller intracranial volume in prodromal Huntington’s disease subjects is reported by Nopoulos et al \cite{Nopoulos2011co}.
\newline

\subsubsection{DWI/DTI studies in HD}

Diffusion-weighted images can non-invasively characterize diffusion in vivo at each voxel \cite{Basser2002}. This makes them useful for investigating of white matter integrity and studying the diseases and abnormalities of white matter of brain. Diffusion at each voxel can be estimated as a tensor in diffusion tensor imaging (DTI). The most common measures derived from DTI data are the rotationally invariant scalars (RISs) that are numeric representations of diffusion shape and magnitude \cite{Basser1996}. White matter integrity measurements derived from RISs can be useful biomarkers of disease progression, since they have been successfully correlated with neuropsychological functioning in both healthy and diseased subjects \cite{Chua2009}.

RISs are applied in whole brain or a region of interest to detect differences between two groups \cite{Snook2007}. Matsui et al. has performed a cross-sectional study comparing mean of RISs across the ROIs between diseased population and controls \cite{Matsui2014}. Different methods, such as voxel-based morphometry \cite{Ashburner2000} and tract-based spatial statistics \cite{Smith2006}, are used to investigate white matter areas of interest.

White matter changes specific to HD has been reported in \cite{Douaud2009}, but it lacks the specificity to link the degradation of a particular pathway to a particular neuropsychological syndrome. Matsui et al. have run a more specific cross-sectional study by examining the diffusivity properties of major WM tracts terminating in prefrontal cortex (PFC) \cite{Matsui2014}.
One of regions of interest in the white matter is cortico-striatal pathway in the brain as key to early HD symptomatology. Neuroimaging studies have shown that white matter connection in the cortico-striatal circuit are affected with HD progression
\cite{Weaver2009, Beglinger2005, Douaud2009}. Studies on areas relevant to the motor loop give consistent information that include changes in diffusion directionality in the posterior limb of internal capsule \cite{Rosas2006, DellaNave2010}, the internal capsule as a whole, and thalamic radiations \cite{Stoffers2010, Bohanna2011}. Frontal lobe has been another targeted area in study of prodromal and symptomatic HD \cite{Reading2005, Rosas2006, DellaNave2010}. Other studies exploring white matter areas have not necessarily been specific to the pathogenesis of HD progression. Other explored white matter regions include corpus callosum \cite{Rosas2006, DellaNave2010, Stoffers2010, Bohanna2011, Weaver2009, Sritharan2010, Muller2011, DiPaola2012, Dumas2012}, corona radiata \cite{DellaNave2010, Stoffers2010, Bohanna2011, Weaver2009}, periventricular white matter \cite{Mascalchi2004}, and whole brain white matter \cite{Rosas2006, Mascalchi2004}.
\newline
